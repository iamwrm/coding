%!TEX TS-program = xelatex 
%!TEX encoding = UTF-8 Unicode

\documentclass[a4paper]{article}

\usepackage{$HOME/Documents/JI/mathdefs}

\newcommand{\rc}{\textrm{c}}

\addtolength{\voffset}{-0.5cm}
\addtolength{\textheight}{3.5cm}       
\addtolength{\topmargin}{-1cm}

\usepackage{booktabs}
\usepackage{engord}
\usepackage{bbm}
\usepackage[colorlinks=true,linkcolor=black]{hyperref}




\renewcommand{\C}{\mathbb{C}}


\renewcommand{\thesection}{Assignment \arabic{section}}
\pagestyle{empty}



\begin{document}
\begin{minipage}[t]{.7\textwidth}
\section*{Vv186 Honors Mathematics II \\ Functions of a Single Variable}
\setcounter{section}{1}
\section{}
\label{ExerciseSet2}
\textbf{Date Due: 8:00 AM, Thursday, the \engordnumber{13} of October 2016}
\end{minipage}\hfill
\begin{minipage}[t]{.3\textwidth}\vspace*{-1Em}\hfill
\mbox{\XeTeXpicfile "$HOME/Documents/JI/ji_logo_2013.png" width .8\textwidth}\vfill
\end{minipage}
\vspace*{1cm}

\emph{Investigation may be likened to the long months of pregnancy, and solving a problem to the day of birth. To investigate a problem is, indeed, to solve it. }\medskip

\hfill Mao Zedong, \emph{Oppose Book Worship}\footnote{{\cn\scriptsize
 调查就像“十月怀胎”,解决问题就像“一朝分娩”。调查就是解决问题。} See \href{http://www.people.com.cn/item/sj/sdldr/mzd/c101.html}{\cn\scriptsize 反对本本主义}. 
Translation taken from the
\href{http://www.marxists.org/reference/archive/mao/selected-works/volume-6/mswv6_11.htm}{Selected Works of Mao Zedong}
at http://www.marxists.org}

\rule{\textwidth}{.1Em}

\begin{Exercise}
	\par\indent\vspace*{0cm}
	\begin{enumerate}
	\item Let $\Q$ denote the rational numbers. Show that the neutral element of addition is unique, i.e., 
	 		if there are two numbers $0,0'\in\Q$ such that $a=a+0=0+a$ and $a=a+0'=0'+a$ for all $a\in\Q$, then 
			$0=0'$.
	\item Show that the inverse element is unique, i.e., if for some $a\in\Q$ there exist $(-a), (-a)'\in\Q$ such that 
			\begin{align*}
			a+(-a)&=(-a)+a=0 & &\text{and} & a+(-a)'&=(-a)'+a=0
			\end{align*}
			then $(-a)=(-a)'$. 
	\end{enumerate}
	\textbf{($\boldsymbol{2+2}$ Marks)}
\end{Exercise}

\begin{Exercise}
\label{sqrttwo.ex}
\par\indent\vspace*{0cm}
\begin{enumerate}
	\item Prove\footnote{See \emph{Spivak}, Ch. 2, Ex. 16.} that if $m,n\in\N^*$ and $m^2/n^2<2$, then $(m+2n)^2/(m+n)^2>2$; 
	show, moreover, that
	\[
	\frac{(m+2n)^2}{(m+n)^2}-2<2-\frac{m^2}{n^2}.
	\]
	\item Prove the same results with all the inequality signs reversed.
	\item Prove that if $(m/n)^2<2$, then there is another rational number 
	$m'/n'$ with $(m/n)^2<(m'/n')^2<2$. (This means that $\max U_1$, 
	\[
	U_1=\{a\in\Q\colon a^2<2\},
	\]
	does not exist in $\Q$.)
	\item Show that $\min U_2$, where $U_2=\{a\in\Q\colon a>0\mathbin{\land} a^2>2\}$, does not exist in $\Q$.
	\item Show that $\inf U_2$ and $\sup U_1$ do not exist in $\Q$.
\end{enumerate}
\textbf{($\boldsymbol{2+2+2+2+2}$ Marks)}
\end{Exercise}

\begin{Exercise}
In the lecture, we proved the uniqueness and the first half of the existence in the theorem
\begin{quote}
	\textbf{Theorem:} For any $x>0$, the set $\R$ contains exactly one positive solution $y$ to the equation $y^2=x$. 
\end{quote}
Complete the proof, i.e., for $y=\inf\{t\in\R\colon t>0\land t^2>x\}$ 
show that the assumption $y^2<x$ leads to a contradiction. \emph{Hint:} use a different strategy than 
the one used to eliminate the possibility that $y^2>x$. If $y^2<x$, show that $(y+\eps)^2<x$ for sufficiently 
small $\eps>0$. How small does $\eps$ have to be? Why does this lead to a contradiction? \par
\textbf{($\boldsymbol{3}$ Marks)}
\end{Exercise}

\begin{Exercise}
Give (without proof) the maximum, minimum, supremum and infimum (if they exist) of the following sets.
\begin{align*}
i)\quad& \left\{ 1+2^{-n}\colon n\in \mathbb{N}\setminus\{0\}\right\}, &
ii)\quad& %\left\{\frac{1}{n}\colon n\in \mathbb{Z}^*\right\}
\Bigl\{ (-1)^n+\frac{1}{n^2}\colon n\in \mathbb{N}\setminus\{0\}\Bigr\}
\end{align*}
\textbf{($\boldsymbol{2+2}$ Marks)}
\end{Exercise}

% \begin{Exercise}
% Let $a,b,c,d\in\R$.  Show that
% \begin{enumerate}
% \item $\frac{a}{b} < \frac{c}{d}$ and $b > 0$, $d > 0$ implies
% $\frac{a}{b} < \frac{a+c}{b+d} < \frac{c}{d}$;
% \item $a > 0$ and $b > 0$ implies $\sqrt{ab} \le
% \frac{a+b}{2}$;
% \item $a > 0$ and $b > 0$ implies $\frac{a}{b} + \frac{b}{a}
% \ge 2$.
% \end{enumerate}
% \textbf{($\boldsymbol{1+1+1}$ Marks)}
% \end{Exercise}

% \begin{Exercise}
% For each of the following inequalities, find the sets of all $x\in\R$ satisfying the inequality.
% \begin{align*}
% \text{i)}\ \abs{x+2} &\le \abs{x -1} & \text{ii)}\ \bigl|2-\abs{x+1}\bigr| &\le 1 &
% % \text{iii)}\, x^3 + 2x^2 - x - 2 &>0
% \end{align*}
% \textbf{($\boldsymbol{2+2}$ Marks)}
% \end{Exercise}





\begin{Exercise}
\label{limsupsets.ex}
A number $x$ is called an \emph{almost upper bound}
 for a set $A\subset\R$ if there are only 
finitely many numbers $y\in A$ with $y\ge x$.\footnote{See \emph{Spivak}, Ch. 8, Ex. 18. ``Finitely many'' means zero or more, but not an infinite number.} An \emph{almost lower bound} is defined similarly. 
\begin{enumerate}
\item 	State (without proof) all amost upper and almost lower bounds for the sets
			\begin{align*}
			\text{a)}\quad& \left\{ 1+2^{-n}\colon n\in \mathbb{N}^*\right\}, &
			\text{b)}\quad& \Bigl\{ (-1)^n+\frac{1}{n^2}\colon n\in \mathbb{N}^*\Bigr\} \\ 
			\text{c)}\quad& \left\{\frac{1}{n}\colon n\in \mathbb{Z}\setminus\{0\}\right\}, &
			\text{d)}\quad& \left\{x\in\Q\colon 0\le x\le \sqrt{2}\right\}
			\end{align*}
\item 	Suppose that $X$ is a bounded infinite set. Prove that the set $Y$ of all almost upper 
		bounds of $X$ is nonempty, and bounded below.
\item 	By (P13), the infimum $\inf Y$ exists; this number is called the \emph{limit superior} of 
		$X$ and denoted by $\limsup X$ or $\varlimsup X$. Find the limit superior for 
		the sets given in i). 
\item 	Formulate a definition for the \emph{limit inferior} 
		$\varliminf X$ and find the limit inferior for 
		the sets given in i).
\item 	Let $A$ be an infinite bounded set.\footnote{See \emph{Spivak}, Ch. 8, Ex. 19. An infinite set is a set with an infinite number of elements.} Prove that
		\begin{enumerate}
			\item $\varliminf A\le \varlimsup A$,
			\item $\varlimsup A\le \sup A$, $\varliminf A\ge \inf A$.
			\item If $\varlimsup A<\sup A$, then $\max A$ exists. If $\varliminf A>\inf A$, then $\min A$ exists.
		\end{enumerate}
\end{enumerate}
\textbf{($\boldsymbol{4+2+2+2+3\times 2}$ Marks)}
\end{Exercise}

\begin{Exercise}\vspace*{0cm}\indent\par
\begin{enumerate}
% \item Let $u = 2 + 3i$, $v = 5 - i$ and $w = 1 + i$. Calculate
% $\abs{u}$, $u + v$, $u - v$, $u \cdot v$, $\frac {u}{v}$ and $u^2 + 2 v w $.
\item For complex numbers $z_1,z_2\in\C$, prove that $\abs{z_1z_2}=\abs{z_1}\abs{z_2}$.
\item Which complex numbers $z\in\C$ satisfy the inequality
$\abs{z + 2} \le \abs{z - 1}$ ?
% \item Calculate $z$ from the equality $(1+2i)z + (1-i)^2 = i-(2+i)z$.
% \item Sketch the following subsets of $\C$:
%       \begin{align*}
%       A &= \{z \in \mathbb{C}\colon \abs{z+1-i}  +  \abs{z-1-i}  = 6\}, &
%       B &= \{z \in \mathbb{C}\colon \abs{z+3} - \abs{z-3}  = 4\}, \\
%       C &= \{z \in \mathbb{C}\colon \abs{z-1-i}  =  \Re(z+1)\}, &
%       D &= \{z \in \mathbb{C}\colon \abs{z-2-3i}  =  4\}.
%       \end{align*}
\item Prove the \emph{parallelogram law} $\abs{z_1+z_2}^2+\abs{z_1-z_2}^2=2(\abs{z_1}^2+\abs{z_2}^2)$, $z_1,z_2\in\C$.
\item Show that in $\C$ it is impossible to define a set $P$ satisfying properties analogous to (P10)-(P12) for real numbers.
\end{enumerate}
\textbf{($\boldsymbol{2+2+2+3}$ Marks)}
\end{Exercise}



\begin{Exercise}
In this exercise, you are supposed to argue directly from the
definition of convergence. Your answers should start with the words
``Let $\eps>0$ be fixed'' and should include an expression for
finding $N$ depending on the given $\eps$.
\begin{enumerate}
\item Show that $\lim\limits_{n\to\infty}\frac{1}{n^2}=0$ and $\lim\limits_{n\to\infty}\frac{2n-5}{n}=2$
\item Let $(a_n), (b_n)$ be complex sequences with limits $a,b \in \C$, respectively.
      Show that
      $\displaystyle\lim_{n \to \infty} a_n b_n = ab$
\end{enumerate}
\textbf{($\boldsymbol{2+2}$ Marks)}
\end{Exercise}


\end{document}